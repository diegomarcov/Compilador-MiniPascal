\documentclass[a4paper,oneside]{article}
\usepackage[spanish]{babel}
\usepackage[latin1]{inputenc}
\usepackage{fullpage}
\usepackage[colorlinks=true,urlcolor=black,linkcolor=black]{hyperref}%
\setlength{\parskip}{1ex plus 0.5ex minus 0.2ex}

\title{Compiladores e Int�rpretes\\Manual de Desarrollo}

\author{Diego Marcovecchio (LU: 83815)\and Leonardo Molas (LU: 82498)}

\date{16 de Septiembre de 2010}
\begin{document}

\maketitle

\section{Casos de prueba}

Los siguientes son los casos de prueba entregados para ser ejecutados junto con el programa, con sus respectivos comportamientos.
\subsection{Casos de prueba correctos}

\subsubsection{\ttfamily min.pas}
Es el m�nimo programa posible.

\subsubsection{\ttfamily boolean.pas}
Se declara un booleano con una expresi�n booleana.

\subsubsection{\ttfamily control.pas}
Se utilizan los distintos tipos de contructores de estructuras de control.

\subsubsection{\ttfamily expression.pas}
Se asigna una variable a una expresi�n aritm�tica compleja.

\subsubsection{\ttfamily programa\_normal.pas}
Se declaran arreglos, procedimiento y funci�n, adem�s de varias asignaciones.

\subsubsection{\ttfamily ejemplo\_full.pas}
El mismo de la entrega anterior, el cual utiliza todos los componenetes del lenguaje.

\subsubsection{\ttfamily declaracion\_subrango.pas}
Se declaran varias variables, tipos, constantes. Se declaran procedimiento y funci�n.


\subsection{Casos de prueba incorrectos}

\subsubsection{\ttfamily caracter\_erroneo.pas}
Este pertenece a los de la entrega anterior. Contiene un error l�xico.

\subsubsection{\ttfamily expresion\_invalida.pas}
Contiene una expresi�n mal formada en la l�nea 21.

\subsubsection{\ttfamily separador\_erroneo.pas}
En la l�nea 22 contiene un punto y coma sobrante en el bloque del if.

\subsubsection{\ttfamily sin\_begin.pas}
Falta el begin en la declaraci�n de la funci�n. En este caso, toma el primer token de la l�nea 8 como parte de la declaraci�n de variables.

\subsubsection{\ttfamily sin\_end.pas}
Falta el end del bloque del programa.

\subsubsection{\ttfamily sin\_program.pas}
Falta el token program en el inicio del programa.

\subsubsection{\ttfamily sin\_puntoycoma.pas}
Falta un punto y coma en la parte de declaraci�n de tipos.

\subsubsection{\ttfamily sin\_tipo.pas}
Falta el tipo de la declaraci�n de un par de variables.

\end{document}