\documentclass[a4paper,oneside]{report}
\usepackage[spanish]{babel}
\usepackage[latin1]{inputenc}
\usepackage{fullpage}
%\usepackage{listings}
%\usepackage{fancyvrb}
\usepackage{float}
\usepackage[colorlinks=true,urlcolor=black,linkcolor=black,citecolor=black]{hyperref}
\usepackage{gmverb}

\setlength{\parskip}{1ex plus 0.5ex minus 0.2ex}

\title{Compiladores e Int�rpretes\\Informe de la Tercera Entrega}

\author{Diego Marcovecchio (LU: 83815)\and Leonardo Molas (LU: 82498)}

\date{16 de Septiembre de 2010}

\begin{document}
	
\maketitle
		
\tableofcontents

\chapter*{Introducci�n}

Esta entrega consiste en el desarrollo del \textbf{Analizador Sint�ctico} que formar� parte del Compilado de Mini-Pascal. Para esto, a partir de un c�digo fuente, se lee la sucesi�n de lexemas, con el Analizador L�xico LexAn, de la entrega anterior, y se verifica que esta sucesi�n sea generada por la gram�tica previamente entregada.

Se documentar� tambi�n la modificaci�n de la gram�tica, de manera tal que pase a ser LL(1).


\chapter{Gram�tica}

\section{Tokens}

En la tabla \ref{tab:tokens} se encuentran todos los tokens con sus respectivos lexemas, como fue presentada en la entrega anterior, con sus debidas modificaciones. 
\begin{table}[htbp]
\begin{tabular}{|l|l|}
\hline
\textbf{Token}			& \textbf{Expresi�n Regular} \\ \hline
Identifier 					& {\ttfamily [a-zA-Z][a-zA-Z0-9]*} \\ \hline
Number							& {\ttfamily [0-9]+} \\ \hline
Char								& {\ttfamily '[a-zA-Z0-9]'} \\ \hline
Less\_Op						& {\ttfamily <\ }  \\ \hline %si se le sacan el backslash espacio se re pincha. es una mierda latex
Greater\_Op					& {\ttfamily >\ } \\ \hline
Greater\_Equal\_Op	& {\ttfamily >=} \\ \hline
Less\_Equal\_Op			& {\ttfamily <=} \\ \hline
Add\_Op		 					& {\ttfamily +}\\ \hline
Minus\_Op 					& {\ttfamily -}\\ \hline
Multiply\_Op				& {\ttfamily *}\\ \hline
Div\_Op							& {\ttfamily [dD][iI][vV]}\\\hline
Not\_LogOp 					& {\ttfamily [nN][oO][tT]} \\ \hline
Or\_LogOp 					& {\ttfamily [oO][rR]} \\ \hline
And\_LogOp 					& {\ttfamily [aA][nN][dD]} \\ \hline
Equal 							& {\ttfamily =} \\ \hline
Type\_Declaration 	& {\ttfamily :} \\ \hline
Assignment 					& {\ttfamily :=} \\ \hline
Comma 							& {\ttfamily ,} \\ \hline
Semicolon 					& {\ttfamily ;} \\ \hline
End\_Program			 	& {\ttfamily .} \\ \hline
Subrange\_Separator & {\ttfamily ..} \\ \hline
Open\_Parenthesis 	& {\ttfamily (} \\ \hline
Close\_Parenthesis 	& {\ttfamily )} \\ \hline
Open\_Bracket 			& {\ttfamily [} \\ \hline
Close\_Bracket 			& {\ttfamily ]} \\ \hline
Program 						& {\ttfamily [pP][rR][oO][gG][rR][aA][mM]} \\ \hline
Type 								& {\ttfamily [tT][yY][pP][eE]} \\ \hline
Const 							& {\ttfamily [cC][oO][nN][sS][tT]} \\ \hline
Var 								& {\ttfamily [vV][aA][rR]} \\ \hline
Function 						& {\ttfamily [fF][uU][nN][cC][tT][iI][oO][nN]} \\ \hline
Procedure 					& {\ttfamily [pP][rR][oO][cC][eE][dD][uU][rR][eE]} \\ \hline
Array 							& {\ttfamily [aA][rR][rR][aA][yY]} \\ \hline
Of 									& {\ttfamily [oO][fF]} \\ \hline
Begin 							& {\ttfamily [bB][eE][gG][iI][nN]} \\ \hline
End 								& {\ttfamily [eE][nN][dD]} \\ \hline
While 							& {\ttfamily [wW][hH][iI][lL][eE]} \\ \hline
Do 									& {\ttfamily [dD][oO]} \\ \hline
If 									& {\ttfamily [iI][fF]} \\ \hline
Then 								& {\ttfamily [tT][hH][eE][nN]} \\ \hline
Else 								& {\ttfamily [eE][lL][sS][eE]} \\ \hline
EOF 								&  \\ \hline
\end{tabular}
\label{tab:tokens}
\caption{Tokens}
\end{table}
\section{Correcci�n de la gram�tica}
Antes de comenzar con la modificaci�n, se corrigieron los errores marcados por la c�tedra. 

\begin{verbatim}
<program> ::= <program heading> <block>.

<program heading> ::= program <identifier>;

<block> ::= <constant definition part><type definition part><variable declaration part><procedure and function declaration part><statement part>

<constant definition part> ::= <empty> | const <constant definition>{;<constant definition>};

<constant definition> ::= <identifier>=<constant>

<identifier> ::= <letter>{<letter or digit>}

<letter or digit> ::= <letter> | <digit>

<letter> ::= A | B | C | D | E | F | G | H | I | J | K | L | M | N | O | P | Q | R | S | T | U | V | W | X | Y | Z | a | b | c | d | e | f | g | h | i | j | k | l | m | n | o | p | q | r | s | t | u | v | w | x | y | z

<digit> ::= 0 | 1 | 2 | 3 | 4 | 5 | 6 | 7 | 8 | 9

<constant> ::= <unsigned number> | <sign><unsigned number> | <constant identifier> | <sign><constant identifier> | <char>

<unsigned number> ::= <unsigned integer>

<unsigned integer> ::= <digit sequence>

<digit sequence> ::= <digit>{<digit>}

<sign> ::= + | -

<constant identifier> ::= <identifier>

<type definition part> ::= <empty> | type <type definition>{;<type definition>};

<type definition> ::= <identifier>=<type>

<type> ::= <simple type> | <structured type>

<simple type> ::= <subrange type> | <type identifier>

<subrange type> ::= <constant>..<constant>

<type identifier> ::= <identifier>

<structured type> ::= <unpacked structured type>

<unpacked structured type> ::= <array type>

<array type> ::= array[<index type>] of <component type>

<index type> ::= <simple type>

<component type> ::= <simple type>

<variable definition part> : := <empty> | var<variable declaration>{;<variable declaration>};

<variable declaration> ::= <identifier>{,<identifier>} : <type>

<procedure and function declaration part> ::= {<procedure or function declaration part>;}

<procedure or function declaration part> ::= <procedure declaration> | <function declaration>

<procedure declaration> ::= <procedure heading><block>

<procedure heading> ::= procedure <identifier>; | procedure <identifier>(<formal parameter section>{;<formal parameter section>});

<formal parameter section> ::= <parameter group> | var <parameter group>

<parameter group> ::= <identifier>{,<identifier>}:<type identifier>

<function declaration> ::= <function heading><block>

<function heading> ::= function<identifier>:<result type>; | <function identifier>(<formal parameter section>{;<formal parameter section>}):<result type>;

<result type> ::= <type identifier>

<statement part> ::= <compound statement>

<compound statement> ::= begin <statement>{;<statement>} end

<statement> ::= <unlabelled statement>

<unlabelled statement> ::= <simple statement> | <structured statement>

<simple statement> ::= <assignment statement> | <procedure statement> | <empty statement>

<assignment statement> ::= <variable>:=<expression> | <function identifier>:=<expression>

<variable> ::= <entire variable> | <component variable>

<entire variable> ::= <variable identifier>

<variable identifier> ::= <identifier>

<component variable> ::= <indexed variable>

<indexed variable> ::= <array variable>[<expression>]

<array variable> ::= <entire variable>

<expression> ::= <simple expression> | <simple expression><relational operator><simple expression>

<simple expression> ::= <term> | <simple expression><adding operator><term> | <sign><term>

<term>::= <factor> | <term><multiplying operator><factor>

<factor> ::= <variable> | <unsigned constant> | <function designator> | (<expression>) | not <factor> | <char>

<char> ::= '<letter>' | '<digit>'

<unsigned constant> ::= <unsigned number> | <constant identifier>

<function designator> ::= <function identifier> | <function identifier>(<actual parameter>{,<actual parameter>})

<function identifier> ::= <identifier>

<actual parameter> ::= <expression> | <variable>

<multiplying operator> ::= * | div | and

<adding operator> ::= + | - | or

<relational operator> ::= = | <> | < | <= | >= | >

<procedure statement> ::= <procedure identifier> | <procedure identifier>(<actual parameter>{,<actual parameter>})

<procedure identifier> ::= <identifier>

<empty statement> ::= <empty>

<structured statement> ::= <compound statement> | <conditional statement> | <repetitive statement>

<conditional statement> ::= <if statement>

<if statement> ::= if <expression> then <statement> | if <expression> then <statement> else <statement>

<repetitive statement> ::= <while statement>

<while statement> ::= while <expression> do <statement>

<special symbol> ::= + | - | * | = | <> | < | > | <= | >= | ( | ) | [ | ] | { | } | := | . | , | ; | : | div | or | and | not | if | then | else | while | do | begin | end | const | var | type | array | function | procedure | program

\end{verbatim}

\section{Pasaje a notaci�n BNF}

Como siguiente paso en la adaptaci�n de la gram�tica, se reemplazaron los terminales por los tokens que devuelve \textbf{LexAn}. Para esto, se adopt� la convenci�n de dejar los no terminales en min�scula, mientras que los tokens (terminales) se encuentran en MAY�SCULA. Luego, se eliminaron las extenciones propias de la notaci�n EBNF.

\begin{verbatim}
<program> ::= <program_heading> <block> <END_PROGRAM>

<program_heading> ::= <PROGRAM> <IDENTIFIER> <SEMI_COLON>

<block> ::= <constant_definition_part> <block_cons_rest> | <block_cons_rest>

<block_cons_rest> ::= <type_definition_part> <block_type_rest> | <block_type_rest>

<block_type_rest> ::= <variable_definition_part> <block_var_rest> | <block_var_rest>

<block_var_rest> ::= <procedure_and_function_declaration_part> <statement_part> | <statement_part>

<constant_definition_part> ::= <CONST> <constant_definition> <constant_definition_rest>

<constant_definition_rest> ::= <SEMI_COLON> <constant_definition_rest_rest>

<constant_definition_rest_rest> ::= <constant_definition> <constant_definition_rest> | <LAMBDA>

<constant_definition> ::= <IDENTIFIER> <EQUAL> <constant>

<constant> ::= <NUMBER> | <IDENTIFIER> | <CHAR> | <sign> <constant_rest>

<constant_rest> ::= <NUMBER> | <IDENTIFIER>

<sign> ::= <ADD_OP> | <MINUS_OP>

<type_definition_part> ::= <TYPE> <type_definition> <type_definition_rest>

<type_definition_rest> ::= <SEMI_COLON> <type_definition_rest_rest>

<type_definition_rest_rest> ::= <type_definition> <type_definition_rest> | <LAMBDA>

<type_definition> ::= <IDENTIFIER> <EQUAL> <type>

<type> ::= <simple_type> | <structured_type>

<simple_type> ::= <NUMBER> <SUBRANGE_SEPARATOR> <constant> | <CHAR> <SUBRANGE_SEPARATOR> <constant> | <sign> <subrange_type_rest> | <IDENTIFIER> <simple_type_rest>

<simple_type_rest> ::= <SUBRANGE_SEPARATOR> <constant> | <LAMBDA>

<subrange_type_rest> ::= <NUMBER> <SUBRANGE_SEPARATOR> <constant> | <IDENTIFIER> <SUBRANGE_SEPARATOR> <constant>

<structured_type> ::= <ARRAY> <OPEN_BRACKET> <simple_type> <CLOSE_BRACKET> <OF> <simple_type>

<variable_definition_part> ::= <VAR> <variable_declaration> <variable_declaration_part_rest>

<variable_declaration_part_rest> ::= <SEMI_COLON> <variable_declaration_rest_rest>

<variable_declaration_rest_rest> ::= <variable_declaration> <variable_declaration_rest> | <LAMBDA>

<variable_declaration> ::= <IDENTIFIER> <variable_declaration_rest>

<variable_declaration_rest> ::= <COMMA> <IDENTIFIER> <variable_declaration_rest> | <TYPE_DECLARATION> <type>

<procedure_and_function_declaration_part> ::= <procedure_or_function_declaration_part> <SEMI_COLON> <procedure_and_function_declaration_part> | <LAMBDA>

<procedure_or_function_declaration_part> ::= <procedure_declaration> | <function_declaration>

<procedure_declaration> ::= <procedure_heading> <block>

<procedure_heading> ::= <PROCEDURE> <IDENTIFIER> <procedure_heading_rest>

<procedure_heading_rest> ::= <SEMI_COLON> | <OPEN_PARENTHESIS> <formal_parameter_section> <formal_parameter_rest>

<formal_parameter_rest> ::= <SEMI_COLON> <formal_parameter_section> <formal_parameter_rest> | <CLOSE_PARENTHESIS> <SEMI_COLON>

<formal_parameter_section> ::= <parameter_group> | <VAR> <parameter_group>

<parameter_group> ::= <IDENTIFIER> <parameter_group_rest>

<parameter_group_rest> ::= <COMMA> <IDENTIFIER> <parameter_group_rest> | <TYPE_DECLARATION> <IDENTIFIER>

<function_declaration> ::= <function_heading> <block>

<function_heading> ::= <FUNCTION> <IDENTIFIER> <function_heading_rest>

<function_heading_rest> ::= <TYPE_DECLARATION> <IDENTIFIER> <SEMI_COLON> | <OPEN_PARENTHESIS> <formal_parameter_section> <formal_parameter_function_rest>

<formal_parameter_function_rest> ::= <SEMI_COLON> <formal_parameter_section> <formal_parameter_function_rest> | <CLOSE_PARENTHESIS> <TYPE_DECLARATION> <IDENTIFIER> <SEMI_COLON>

<statement_part> ::= <BEGIN> <statement_part_rest>

<statement_part_rest> ::= <statement> <statement_rest> | <statement_rest>

<statement_rest> ::= <SEMI_COLON> <statement_rest_rest> | <END>

<statement_rest_rest> ::= <statement> <statement_rest> | <statement_rest>

<statement> ::= <simple_statement> | <structured_statement>

<simple_statement> ::= <IDENTIFIER> <simple_statement_rest>

<simple_statement_rest> ::= <ASSIGNMENT> <expression> | <OPEN_BRACKET> <expression> <CLOSE_BRACKET> <ASSIGNMENT> <expression> | <OPEN_PARENTHESIS> <actual_parameter> <actual_parameter_rest> | <LAMBDA>

<component_variable> ::= <IDENTIFIER> <OPEN_BRACKET> <expression> <CLOSE_BRACKET>

<expression> ::= <simple_expression> <expression_rest>

<expression_rest> ::= <relational_operator> <simple_expression> | <LAMBDA>

<simple_expression> ::= <term> <simple_expression_other> | <sign> <term> <simple_expression_other>

<simple_expression_other> ::= <adding_operator> <term> <simple_expression_other> | <LAMBDA>

<term> ::= <factor> <term_other>

<term_other> ::= <multiplying_operator> <factor> <term_other> | <LAMBDA>

<factor> ::= <IDENTIFIER> <factor_rest> | <NUMBER> | <OPEN_PARENTHESIS> <expression> <CLOSE_PARENTHESIS> | <NOT_LOGOP> <factor> | <CHAR>

<factor_rest> ::= <OPEN_BRACKET> <expression> <CLOSE_BRACKET> | <OPEN_PARENTHESIS> <actual_parameter> <actual_parameter_rest> | <LAMBDA>

<actual_parameter> ::= <expression>

<actual_parameter_rest> ::= <COMMA> <actual_parameter> <actual_parameter_rest> | <CLOSE_PARENTHESIS>

<multiplying_operator> ::= <MULTIPLY_OP> | <DIV_OP> | <AND_LOGOP>

<adding_operator> ::= <ADD_OP> | <MINUS_OP> | <OR_LOGOP>

<relational_operator> ::= <LESS_OP> | <LESS_EQUAL_OP> | <GREATER_OP> | <GREATER_EQUAL_OP> | <EQUAL>

<procedure_statement> ::= <IDENTIFIER> <procedure_statement_rest>

<procedure_statement_rest> ::= <OPEN_PARENTHESIS> <actual_parameter> <actual_parameter_rest> | <LAMBDA>

<structured_statement> ::= <BEGIN> <structured_statement_other> | <conditional_statement> | <repetitive_statement>

<structured_statement_other> ::= <statement> <statement_rest> | <statement_rest>

<conditional_statement> ::= <IF> <expression> <THEN> <statement> <conditional_statement_rest>

<conditional_statement_other> ::= <ELSE> <statement> | <LAMBDA>

<repetitive_statement> ::= <WHILE> <expression> <DO> <repetitive_statement_rest>

<repetitive_statement_rest> ::= <statement> | <LAMBDA>
\end{verbatim}

\section{Gram�tica final}
Para llegar a la gram�tica utilizada para implementar el analizador l�xico, se realizaron varios pasos:
\begin{enumerate}
	\item \textbf{Eliminar Ambiguedad}: Este tal vez sea la afirmaci�n m�s peligrosa, ya que no se puede saber si una gram�tica es ambigua o no. De cualquier manera, se eliminaron todas las que se encontraron, salvo el caso del \verb|if then else|, que se resolver� en el an�lisis sem�ntico.
	\item \textbf{Eliminar Recursi�n a Izquierda}: Se utiliz� el algoritmo explicado en \cite[p�g. 212]{aho}.
	\item \textbf{Factorizar a Izquierda}: Se utiliz� el algoritmo explicado en el mismo libro, en la p�gina 214.
\end{enumerate}
Luego de esta serie de pasos, se lleg� a la siguiente gram�tica:

\begin{verbatim}

\end{verbatim}

\begin{thebibliography}{9}

\bibitem{aho}
  Alfred V. Aho, Monica S. Lam, Ravi Sethi, Jeffrey D. Ullman
  \emph{Compilers: principles, techniques, and tools}.
  Addison Wesley
  2nd Edition
  2007.

\end{thebibliography}

\end{document}