\documentclass[a4paper,oneside]{report}
\usepackage[spanish]{babel}
\usepackage[latin1]{inputenc}
\usepackage{fullpage}
%\usepackage{listings}
%\usepackage{fancyvrb}
\usepackage{float}
\usepackage[colorlinks=true,urlcolor=black,linkcolor=black,citecolor=black]{hyperref}
%\usepackage{gmverb}

\setlength{\parskip}{1ex plus 0.5ex minus 0.2ex}

\title{Compiladores e Intérpretes\\Manual del Usuario}

\author{Diego Marcovecchio (LU: 83815)\and Leonardo Molas (LU: 82498)}

\date{2 de Diciembre de 2010}

\begin{document}
	
\maketitle
		
\tableofcontents

\chapter*{Introducción}
\addcontentsline{toc}{chapter}{Introducción}

Este es el Manual del Usuario del Compilador de Mini-Pascal \textsc{pyComp}. Detallaremos el alcance del compilador, el lenguaje y los símbolos utilizados, los tipos de datos cubiertos; la forma de uso del lenguaje, la utilización en general del compilador, y las decisiones de diseño tomadas.

\chapter{Modo de Uso}

\section{Requerimientos}
Para ejecutar el compilador es necesario contar con las siguientes librerías (que pueden encontrarse en la carpeta entregada a la cátedra).

\begin{itemize}\ttfamily
	\item python27.dll
	\item msvcr90.dll
	\item bz2.pyd
	\item select.pyd
	\item unicodedata.pyd
	\item library.zip \rmfamily (que contiene las librerías de Python utilizadas)
\end{itemize}

\section{Ejecución}

\verb|pyComp IN_FILE [OUT_FILE] [-h] [-d] [-o DISPLAY_FILE]|

\subsection*{Argumentos}

\begin{description}
	\item[{\ttfamily IN\_FILE}] El archivo de Pascal de entrada.
\end{description}

\subsection*{Argumentos opcionales}

\begin{description}
	\item[\ttfamily OUT\_FILE] El archivo opcional de salida. En caso de especificarse, en éste archivo se generará el código MEPa correspondiente a {\ttfamily IN\_FILE}; en caso de no especificarse, se removerán los últimos tres caracteres del {\ttfamily IN\_FILE}, y se reemplazarán por \"mepa\", creando un nuevo archivo con ese nombre.
	\item[\ttfamily -d] Modo \textit{debug} (utilizado durante el desarrollo, y dejado por la posible utilidad en el futuro).
	\item[\ttfamily -h, --help] Muestra la ayuda por pantalla.
	\item[\ttfamily -o DISPLAY\_FILE] Hace que la salida del compilador se muestre en el archivo {\ttfamily DISPLAY\_FILE}. Por defecto, {\ttfamily DISPLAY\_FILE} es el archivo de salida standard del sistema operativo, por lo que de no especificarse, la salida será realizada por la pantalla.
\end{description}

\section{Formato de la salida}
La salida por pantalla de \emph{pyComp} mostrará un mensaje de éxito si el programa es correcto, o bien un mensaje de error indicando el tipo de error, y el número de línea en el que fue detectado.

Los errores posiblemente devueltos son:

\begin{itemize}
  \item \emph{Lexical error}: ocurre cuando el compilador se topa con un símbolo desconocido. La lista de símbolos válidos puede ser encontrada en el capítulo correspondiente a \textbf{Lenguaje}.
  \item \emph{Syntactical error}: ocurre cuando el compilador encuentra una sentencia que viola la gramática definida.
  \item \emph{Semantical error}: ocurre cuando se viola alguna de las reglas semánticas definidas por el compilador; por ejemplo, cuando se referencia a una variable que no fue definida anteriormente, cuando se asignan dos elementos de tipos incompatibles, o cuando se declaran dos variables con el mismo identificador.
\end{itemize}

Por añadidura, se informará por pantalla en forma de \texttt{warning} cuando haya variables, funciones o procedimientos declarados que no hayan sido utilizados.

\begin{figure}[h]
\begin{verbatim}
Starting file lexical and syntactical analysis...


The program is syntactically correct.
\end{verbatim}
\caption{Salida por pantalla de un programa sintácticamente correcto}
\label{fig:correct}
\end{figure}



\begin{figure}[h]
\begin{verbatim}
Starting file lexical and syntactical analysis...



"bateria\ejemplo10.pas", line 9: Syntactical error found: Expecting "," or ":",
	but ":=" was found
\end{verbatim}
\caption{Salida por pantalla de un programa con un error sintáctico}
\label{fig:error}
\end{figure}



\begin{figure}[h]
\begin{verbatim}
Starting file lexical and syntactical analysis...


"bateria\unusedIdentifier.pas", line 1: WARNING: 'c' has never been initialized
	in program 'unusedidentifier'

The program is syntactically correct.
\end{verbatim}
\caption{Salida por pantalla de un programa correcto con una variable no utilizada}
\label{fig:warning}
\end{figure}

\chapter{Especificación completa del lenguaje}

\section{Análisis léxico}

En esta sección definiremos todos los símbolos válidos en el lenguaje Mini-Pascal. Éstos se dividen en letras, números, y símbolos especiales (dentro de los que se consideran también las palabras reservadas).

\begin{verbatim}
<letter> ::= A | B | C | D | E | F | G | H | I | J | K | L | M | N | O | P | Q | R | S | T | U | V | W | X | Y | Z | a | b | c | d | e | f | g | h | i | j | k | l | m | n | o | p | q | r | s | t | u | v | w | x | y | z

<digit> ::= 0 | 1 | 2 | 3 | 4 | 5 | 6 | 7 | 8 | 9

<special symbol> ::= + | - | * | = | <> | < | > | <= | >= | ( | ) | [ | ] | { | } | := | . | , | ; | : | div | or | and | not | if | then | else | while | do | begin | end | const | var | type | array | function | procedure | program
\end{verbatim}

\subsection{Comentarios}

Los comentarios son bloques de texto que se encuentran en el source del programa, pero son ignorados por el analizador léxico y no generan código intermedio. Pueden contener cualquier tipo de símbolos, pertenezcan al lenguaje o no.

\textsc{pycomp} utiliza los siguientes:

\begin{itemize}
	\item \textbf{Comentarios de línea}: la doble barra (``//'') comienza un comentario que termina en el fin de línea.
	\item \textbf{Comentarios de bloque}: los comentarios de bloque comienzan cuando se detecta el primero de los símbolos, y terminan al encontrar el segundo; pueden abarcar cualquier cantidad de líneas. Es importante destacar que un comentario abierto, pero no cerrado, produce un error léxico. Hay dos maneras de marcar los comentarios de bloque: \verb|{ y }|, o \verb|(* y *)|.
\end{itemize}

Puede verse un ejemplo de la utilización de los comentarios en la figura \ref{comentarios}.

\begin{figure}[h]
\begin{verbatim}
{ 
 Descripcion del programa : Este programa muestra un mensaje por pantalla
}
PROGRAM Saludo;
BEGIN
  Write('H') // Instruccion que muestra el mensaje por pantalla 
END.

(* Aquí termina el programa *)
\end{verbatim}
\caption{Ejemplo del uso de comentarios}
\label{comentarios}
\end{figure}

\subsection{Identificadores}

Un identificador es utilizado para dar nombre a un programa, constante, tipo, variable, función o procedimiento.
\begin{verbatim}
<identifier> ::= <letter>{<letter or digit>}
\end{verbatim}

\subsection{Números}
Un número es una secuencia de dígitos decimales.
\begin{verbatim}
<digit sequence> ::= <digit>{<digit>}
\end{verbatim}

\section{Declaraciones}

\subsection{Constantes}
Cuando se define una constante, el identificador utilizado se liga al valor elegido durante toda la ejecución del programa; dicho valor no podrá ser cambiado.

\begin{verbatim}
<constant definition> ::= <identifier>=<constant>
<constant> ::= <unsigned number> | <sign><unsigned number> | <constant identifier> | <sign><constant identifier> | <char>
<constant identifier> ::= <identifier>
<sign> ::= + | -
\end{verbatim}

\subsection{Tipos}

Los tipos determinan los valores posibles que puede tener una variable en Mini-Pascal, así como las posibles operaciones que se puedan aplicar sobre este. Se asocian a un identificador.

\begin{verbatim}
<type definition> ::= <identifier>=<type>

<type> ::= <simple type> | <structured type>
\end{verbatim}


\subsubsection{Tipos Simples}

Dentro de los tipos, se encuentran los tipos simples, que cumplen con la característica de ocupar sólo una celda de memoria. Hay tres tipos simples predefinidos en Mini-Pascal, que serán detallados luego en la sección pertinente.

A su vez, se encuentran los \textit{subrangos}, que definen cotas superiores e inferiores a partir de constantes, cuyos tipos deben coincidir en el tipo del subrango.

\begin{verbatim}
<simple type> ::= <subrange type> | <type identifier>

<subrange type> ::= <constant>..<constant>

<type identifier> ::= <identifier>
\end{verbatim}

\subsubsection{Tipos Compuestos}

El único tipo compuesto posible dentro de Mini-Pascal es el arreglo: una serie de locaciones de memoria referenciados por índices correlativos. Éste debe tener tipos simples tanto de elementos como de índices.
\begin{verbatim}
<structured type> ::= <unpacked structured type>

<unpacked structured type> ::= <array type>

<array type> ::= array[<index type>] of <component type>

<index type> ::= <simple type>

<component type> ::= <simple type>
\end{verbatim}

\subsection{Variables}

Las variables son una abstracción de referencias a celdas de memoria, cuyo valor puede ser cambiado en tiempo de ejecución. Tienen un tipo determinado en tiempo de compilación, y están asociados a un identificador.

\begin{verbatim}
<variable declaration> ::= <identifier>{,<identifier>} : <type>
\end{verbatim}

\subsection{Procedimientos}

Los procedimientos pueden pensarse como subprogramas, determinados por un identificador que se utiliza para luego invocarlo.

\begin{verbatim}
<procedure declaration> ::= <procedure heading><block>
\end{verbatim}

\subsubsection{Encabezado de procedimiento}

Los procedimentos son asociados a un identificador, y pueden tener una serie de parámetros, que se determinan en tiempo de compilación.

Un grupo de parámetros que tengan antepuesta la palabra \texttt{var} serán tratados como \textit{parámatros por refencia}; es decir, no se copiarán los valores en nuevas locaciones de memoria, sino que se utilizará la misma locación. Ésto puede generar efectos colaterales.

Los parámetros que no sean por referencia, son denominados \textit{parámtros por valor}. Se trata de variables que sólo vivirán mientras se ejecute el procedimiento, y que serán inicializadas a partir del valor que les sea pasado en la llamada.

\begin{verbatim}
<procedure heading> ::= procedure <identifier>; | procedure <identifier>(<formal parameter section>{;<formal parameter section>});

<formal parameter section> ::= <parameter group> | var <parameter group>

<parameter group> ::= <identifier>{,<identifier>}:<type identifier>
\end{verbatim}
\subsection{Funciones}

Las funciones son muy parecidas a los procedimientos, con la diferencia que devuelven un valor, por lo que no son utilizadas como sentencias de por sí, sino que se invocan dentro del contexto de una expresión.

\begin{verbatim}
<function declaration> ::= <function heading><block>
\end{verbatim}

\subsubsection{Encabezado de funciones}

Así como los procedimientos, las funciones están asociadas a un identificador, y tienen una signatura determinada en tiempo de compilación. Puede tener parámatros, con sus tipos, y debe tener un tipo de resultado, que será el tipo del valor que devolverá. En Mini-Pascal, sólo se aceptarán como tipos de resultado, tipos simples.

Puede tener parámtros por valor o referencia, como los procedimientos.

\begin{verbatim}
<function heading> ::= function<identifier>:<result type>; | <function identifier>(<formal parameter section>{;<formal parameter section>}):<result type>;

<result type> ::= <type identifier>
\end{verbatim}

\section{Expresiones}

Las expresiones denotan la computación a llevar a cabo en una determinada sentencia. Se computan a partir de átomos, y generan un valor de salida, luego de haber aplicado todos los operadores, con su determinado tipo.

\subsection{Operadores}

Los operadores que se detallarán son binarios, que reciben dos argumentos del mismo tipo, y generan un nuevo valor a partir de ellas. 

\begin{verbatim}
<multiplying operator> ::= * | div | and

<adding operator> ::= + | - | or

<relational operator> ::= = | <> | < | <= | >= | >
\end{verbatim}


\section{Sentencias Simples}

Las sentencias simples componen los bloques de programa en el lenguaje Mini-Pascal. Sólo se contará con asignaciones, o llamadas a procedimientos. A su vez, una sentencia simple puede ser vacía.



\begin{verbatim}
<simple statement> ::= <assignment statement> | <procedure statement> | <empty statement>

<empty statement> ::= <empty>
\end{verbatim}

\subsection{Asignaciones}
Las asignaciones son sentencias que modifican el valor de una variable, o de un componente de una variable de tipo complejo. Para esto, se calcula el valor de una expresión, cuyo tipo debe coincidir con el tipo de la variable.

Existe el caso especial de la asignación de valores de retorno de una función. En este caso, en vez de utilizar el nombre de una variable al principio de una asignación, se pone el nombre de la función del mismo nivel léxico. El valor que retorne la expresión será el valor que devuelva la función, en el lugar donde haya sido llamada.

\begin{verbatim}
<assignment statement> ::= <variable>:=<expression> | <function identifier>:=<expression>

<variable> ::= <entire variable> | <component variable>

<entire variable> ::= <variable identifier>

<variable identifier> ::= <identifier>

<component variable> ::= <indexed variable>

<indexed variable> ::= <array variable>[<expression>]

<array variable> ::= <entire variable>
\end{verbatim}

\subsection{Llamadas a procedimiento}

Los procedimientos son llamados anteponiendo su identificador antes de la lista de parámetros actuales. Estos últimos van entre paréntesis, y separados por comas. Deben mandarse la misma cantidad de parámetros que tenga el procedimento invocado, y cada uno debe tener el mismo tipo que el parámetro formal con la misma posición.

Los parámetros actuales son expresiones, a las que se les calcula el valor resultante, y éste será el valor que tomará el parámetro que tenga la misma posición.

Existe el caso especial de los parámetros por referencia, en el cual los parámetros actuales deberán ser variables, o componentes de variables de tipo complejo. Dicha variable puede cambiar de valor al terminar el procedimiento, dado que si el parámetro se modifica dentro del procedimiento, este cambio se mantendrá terminada su ejecución.

\begin{verbatim}
<procedure statement> ::= <procedure identifier> | <procedure identifier>(<actual parameter>{,<actual parameter>})

<procedure identifier> ::= <identifier>
\end{verbatim}



\section{Sentencias Compuestas}

Las sentencias compuestas pueden ser una suceción de otras sentencias, agrupadas entre delimitadores; sentencias condicionales (\texttt{})

\begin{verbatim}
<structured statement> ::= <compound statement> | <conditional statement> | <repetitive statement>
\end{verbatim}

\section{Declaraciones}


\section{Programas}


\subsection{Tipos predefinidos}
Los tipos predefinidos en Mini-Pascal son \texttt{char} (caracteres), \texttt{integer} (números enteros) y \texttt{boolean} (\textbf{true} y \textbf{false}).

\section{Corrección de la gramática}
Antes de comenzar con la modificación, se corrigieron los errores marcados por la cátedra.
\begin{verbatim}
<program> ::= <program heading> <block>.

<program heading> ::= program <identifier>;

<block> ::= <constant definition part><type definition part><variable declaration part><procedure and function declaration part><statement part>

<constant definition part> ::= <empty> | const <constant definition>{;<constant definition>};

<constant definition> ::= <identifier>=<constant>

<identifier> ::= <letter>{<letter or digit>}

<letter or digit> ::= <letter> | <digit>

<letter> ::= A | B | C | D | E | F | G | H | I | J | K | L | M | N | O | P | Q | R | S | T | U | V | W | X | Y | Z | a | b | c | d | e | f | g | h | i | j | k | l | m | n | o | p | q | r | s | t | u | v | w | x | y | z

<digit> ::= 0 | 1 | 2 | 3 | 4 | 5 | 6 | 7 | 8 | 9

<constant> ::= <unsigned number> | <sign><unsigned number> | <constant identifier> | <sign><constant identifier> | <char>

<unsigned number> ::= <unsigned integer>

<unsigned integer> ::= <digit sequence>

<digit sequence> ::= <digit>{<digit>}

<sign> ::= + | -

<constant identifier> ::= <identifier>

<type definition part> ::= <empty> | type <type definition>{;<type definition>};

<type definition> ::= <identifier>=<type>

<type> ::= <simple type> | <structured type>

<simple type> ::= <subrange type> | <type identifier>

<subrange type> ::= <constant>..<constant>

<type identifier> ::= <identifier>

<structured type> ::= <unpacked structured type>

<unpacked structured type> ::= <array type>

<array type> ::= array[<index type>] of <component type>

<index type> ::= <simple type>

<component type> ::= <simple type>

<variable definition part> : := <empty> | var<variable declaration>{;<variable declaration>};

<variable declaration> ::= <identifier>{,<identifier>} : <type>

<procedure and function declaration part> ::= {<procedure or function declaration part>;}

<procedure or function declaration part> ::= <procedure declaration> | <function declaration>

<procedure declaration> ::= <procedure heading><block>

<procedure heading> ::= procedure <identifier>; | procedure <identifier>(<formal parameter section>{;<formal parameter section>});

<formal parameter section> ::= <parameter group> | var <parameter group>

<parameter group> ::= <identifier>{,<identifier>}:<type identifier>

<function declaration> ::= <function heading><block>

<function heading> ::= function<identifier>:<result type>; | <function identifier>(<formal parameter section>{;<formal parameter section>}):<result type>;

<result type> ::= <type identifier>

<statement part> ::= <compound statement>

<compound statement> ::= begin <statement>{;<statement>} end

<statement> ::= <unlabelled statement>

<unlabelled statement> ::= <simple statement> | <structured statement>

<simple statement> ::= <assignment statement> | <procedure statement> | <empty statement>

<assignment statement> ::= <variable>:=<expression> | <function identifier>:=<expression>

<variable> ::= <entire variable> | <component variable>

<entire variable> ::= <variable identifier>

<variable identifier> ::= <identifier>

<component variable> ::= <indexed variable>

<indexed variable> ::= <array variable>[<expression>]

<array variable> ::= <entire variable>

<expression> ::= <simple expression> | <simple expression><relational operator><simple expression>

<simple expression> ::= <term> | <simple expression><adding operator><term> | <sign><term>

<term>::= <factor> | <term><multiplying operator><factor>

<factor> ::= <variable> | <unsigned constant> | <function designator> | (<expression>) | not <factor> | <char>

<char> ::= '<letter>' | '<digit>'

<unsigned constant> ::= <unsigned number> | <constant identifier>

<function designator> ::= <function identifier> | <function identifier>(<actual parameter>{,<actual parameter>})

<function identifier> ::= <identifier>

<actual parameter> ::= <expression> | <variable>

<multiplying operator> ::= * | div | and

<adding operator> ::= + | - | or

<relational operator> ::= = | <> | < | <= | >= | >

<procedure statement> ::= <procedure identifier> | <procedure identifier>(<actual parameter>{,<actual parameter>})

<procedure identifier> ::= <identifier>

<empty statement> ::= <empty>

<structured statement> ::= <compound statement> | <conditional statement> | <repetitive statement>

<conditional statement> ::= <if statement>

<if statement> ::= if <expression> then <statement> | if <expression> then <statement> else <statement>

<repetitive statement> ::= <while statement>

<while statement> ::= while <expression> do <statement>

<special symbol> ::= + | - | * | = | <> | < | > | <= | >= | ( | ) | [ | ] | { | } | := | . | , | ; | : | div | or | and | not | if | then | else | while | do | begin | end | const | var | type | array | function | procedure | program

\end{verbatim}


\section{Gramática final}
Para llegar a la gramática utilizada para implementar el analizador sintáctico, se realizaron varios pasos:
\begin{enumerate}
	\item \textbf{Eliminar Ambiguedad}: ésta tal vez sea la afirmación más peligrosa, ya que no se puede saber si una gramática es ambigua o no. De cualquier manera, se eliminaron todas las ambiguedades que se encontraron, salvo el caso del \verb|if then else|, del cual se hablará más adelante.
	\item \textbf{Eliminar Recursión a Izquierda}: Se utilizó el algoritmo explicado en \cite[pág. 212]{aho}.
	\item \textbf{Factorizar a Izquierda}: Se utilizó el algoritmo explicado en el mismo libro, en la página 214.
\end{enumerate}
Luego de esta serie de pasos, se llegó a la siguiente gramática:

\begin{verbatim}
<program> ::= <program_heading> <block> <END_PROGRAM> <EOF>

<program_heading> ::= <PROGRAM> <IDENTIFIER> <SEMI_COLON>

<block> ::= <constant_definition_part> <block_cons_rest> | <block_cons_rest>

<block_cons_rest> ::= <type_definition_part> <block_type_rest> | <block_type_rest>

<block_type_rest> ::= <variable_definition_part> <block_var_rest> | <block_var_rest>

<block_var_rest> ::= <procedure_and_function_declaration_part> <statement_part> | <statement_part>

<constant_definition_part> ::= <CONST> <constant_definition> <constant_definition_rest>

<constant_definition_rest> ::= <SEMI_COLON> <constant_definition_rest_rest>

<constant_definition_rest_rest> ::= <constant_definition> <constant_definition_rest> | <LAMBDA>

<constant_definition> ::= <IDENTIFIER> <EQUAL> <constant>

<constant> ::= <NUMBER> | <IDENTIFIER> | <CHAR> | <sign> <constant_rest>

<constant_rest> ::= <NUMBER> | <IDENTIFIER>

<sign> ::= <ADD_OP> | <MINUS_OP>

<type_definition_part> ::= <TYPE> <type_definition> <type_definition_rest>

<type_definition_rest> ::= <SEMI_COLON> <type_definition_rest_rest>

<type_definition_rest_rest> ::= <type_definition> <type_definition_rest> | <LAMBDA>

<type_definition> ::= <IDENTIFIER> <EQUAL> <type>

<type> ::= <simple_type> | <structured_type>

<simple_type> ::= <NUMBER> <SUBRANGE_SEPARATOR> <constant> | <CHAR> <SUBRANGE_SEPARATOR> <constant> | <sign> <subrange_type_rest> | <IDENTIFIER> <simple_type_rest>

<simple_type_rest> ::= <SUBRANGE_SEPARATOR> <constant> | <LAMBDA>

<subrange_type_rest> ::= <NUMBER> <SUBRANGE_SEPARATOR> <constant> | <IDENTIFIER> <SUBRANGE_SEPARATOR> <constant>

<structured_type> ::= <ARRAY> <OPEN_BRACKET> <simple_type> <CLOSE_BRACKET> <OF> <simple_type>

<variable_definition_part> ::= <VAR> <variable_declaration> <variable_declaration_part_rest>

<variable_declaration_part_rest> ::= <SEMI_COLON> <variable_declaration_rest_rest>

<variable_declaration_rest_rest> ::= <variable_declaration> <variable_declaration_part_rest> | <LAMBDA>

<variable_declaration> ::= <IDENTIFIER> <variable_declaration_rest>

<variable_declaration_rest> ::= <COMMA> <IDENTIFIER> <variable_declaration_rest> | <TYPE_DECLARATION> <type>

<procedure_and_function_declaration_part> ::= <procedure_or_function_declaration_part> <SEMI_COLON> <procedure_and_function_declaration_part> | <LAMBDA>

<procedure_or_function_declaration_part> ::= <procedure_declaration> | <function_declaration>

<procedure_declaration> ::= <procedure_heading> <block>

<procedure_heading> ::= <PROCEDURE> <IDENTIFIER> <procedure_heading_rest>

<procedure_heading_rest> ::= <SEMI_COLON> | <OPEN_PARENTHESIS> <formal_parameter_section> <formal_parameter_rest>

<formal_parameter_rest> ::= <SEMI_COLON> <formal_parameter_section> <formal_parameter_rest> | <CLOSE_PARENTHESIS> <SEMI_COLON>

<formal_parameter_section> ::= <parameter_group> | <VAR> <parameter_group>

<parameter_group> ::= <IDENTIFIER> <parameter_group_rest>

<parameter_group_rest> ::= <COMMA> <IDENTIFIER> <parameter_group_rest> | <TYPE_DECLARATION> <IDENTIFIER>

<function_declaration> ::= <function_heading> <block>

<function_heading> ::= <FUNCTION> <IDENTIFIER> <function_heading_rest>

<function_heading_rest> ::= <TYPE_DECLARATION> <IDENTIFIER> <SEMI_COLON> | <OPEN_PARENTHESIS> <formal_parameter_section> <formal_parameter_function_rest>

<formal_parameter_function_rest> ::= <SEMI_COLON> <formal_parameter_section> <formal_parameter_function_rest> | <CLOSE_PARENTHESIS> <TYPE_DECLARATION> <IDENTIFIER> <SEMI_COLON>

<statement_part> ::= <compound_statement>

<compound_statement> ::= <BEGIN> <statement> <statement_part_rest> <END>

<statement_part_rest> ::= <SEMI_COLON> <statement> <statement_part_rest> | <LAMBDA>

<statement> ::= <simple_statement> | <structured_statement>

<simple_statement> ::= <IDENTIFIER> <simple_statement_rest> | <LAMBDA>

<simple_statement_rest> ::= <ASSIGNMENT> <expression> | <OPEN_BRACKET> <expression> <CLOSE_BRACKET> <ASSIGNMENT> <expression> | <OPEN_PARENTHESIS> <actual_parameter> <actual_parameter_rest> | <LAMBDA>

<component_variable> ::= <IDENTIFIER> <OPEN_BRACKET> <expression> <CLOSE_BRACKET>

<expression> ::= <simple_expression> <expression_rest>

<expression_rest> ::= <relational_operator> <simple_expression> | <LAMBDA>

<simple_expression> ::= <term> <simple_expression_other> 

<simple_expression_other> ::= <adding_operator> <term> <simple_expression_other> | <LAMBDA>

<term> ::= <factor> <term_other>

<term_other> ::= <multiplying_operator> <factor> <term_other> | <LAMBDA>

<factor> ::= <IDENTIFIER> <factor_rest> | <NUMBER> | <OPEN_PARENTHESIS> <expression> <CLOSE_PARENTHESIS> | <NOT_LOGOP> <factor> | <CHAR> | <sign> <factor>

<factor_rest> ::= <OPEN_BRACKET> <expression> <CLOSE_BRACKET> | <OPEN_PARENTHESIS> <actual_parameter> <actual_parameter_rest> | <LAMBDA>

<actual_parameter> ::= <expression>

<actual_parameter_rest> ::= <COMMA> <actual_parameter> <actual_parameter_rest> | <CLOSE_PARENTHESIS>

<multiplying_operator> ::= <MULTIPLY_OP> | <DIV_OP> | <AND_LOGOP>

<adding_operator> ::= <ADD_OP> | <MINUS_OP> | <OR_LOGOP>

<relational_operator> ::= <LESS_OP> | <LESS_EQUAL_OP> | <GREATER_OP> | <GREATER_EQUAL_OP> | <NOT_EQUAL_OP> | <EQUAL>

<structured_statement> ::= <compound_statement> | <conditional_statement> | <repetitive_statement>

<conditional_statement> ::= <IF> <expression> <THEN> <statement> <conditional_statement_other>

<conditional_statement_other> ::= <ELSE> <statement> | <LAMBDA>

<repetitive_statement> ::= <WHILE> <expression> <DO> <statement>
\end{verbatim}

\section{¿Es LL(1)?}
La gramática pasó por todos los pasos especificados, en un intento de lograr una gramática para ser utilizada como base de la implementación de un Analizador Sintáctico Descendente Predictivo Recursivo. Cada regla tiene conjuntos disjuntos de la función \textsc{Primero} para cada producción, por lo que el analizador puede saber que producción optar, leyendo un token. Por esto, se podría decir que la gramática obtenida es LL(1).

Sin embargo, como fue dicho en la sección anterior, la gramática contiene por lo menos una ambigüedad: la del \verb|if then else|. La cadena de tokens \verb|<IF> ... <THEN> <IF> ... <THEN> ... <ELSE> ...| (donde los puntos suspensivos representan un grupo de statements válidos) tiene dos árboles de derivación posibles. 

Una solución que se puede adoptar es la de utilizar precedencia, donde el \verb|<ELSE>| quedará ligado al \verb|<IF> ... <THEN>| sin \verb|<ELSE>| más cercano. Ésta solución será la que se adopte para la implementación del compilador de \textsc{Mini-Pascal}, y se llevará a cabo en la etapa de análisis semántico.

Se puede concluir entonces que la gramática no es LL(1), pero de cualquier manera, sirve para realizar el Analizador Sintáctico.

\chapter{Errores detectados}
Dado las características del Analizador Sintáctico solicitado por la cátedra, los errores detectados son disparados cuando se encuentra un \textit{token} distinto al que se esperaba. Por esto, el mensaje impreso por pantalla (o en el archivo de salida) muestra el dicho \textit{token}.

Muchas veces, está información no es suficiente para el programador, por lo que, en la mayoría de los posibles errores, también se detalla cuál era el token (o los tokens) esperados, como se muestra en la figura \ref{fig:error}.


\begin{figure}[h]
\begin{verbatim}
Starting file lexical and syntactical analysis...



"bateria\ejemplo5.pas", line 22: Unexpected token: ")" found.
\end{verbatim}
\label{fig:error2}
\caption{Error posible: Token inesperado}
\end{figure}

\begin{thebibliography}{9}
\addcontentsline{toc}{chapter}{Bibliografía}
\bibitem{aho}
  Alfred V. Aho, Monica S. Lam, Ravi Sethi, Jeffrey D. Ullman
  \emph{Compilers: principles, techniques, and tools}.
  Addison Wesley
  2nd Edition
  2007.
  
\bibitem{enunciado}
	Cátedra de Compiladores e Intérpretes, DCIC, UNS
	\emph{Proyecto Nº1: Compilador de Mini-Pascal-S}
	2010
	
\bibitem{consideraciones}
	Cátedra de Compiladores e Intérpretes, DCIC, UNS
	\emph{Consideraciones Generales para la 3er entrega del Proyecto Nº1 - Analizador Sintáctico}
	2010

\end{thebibliography}

\end{document}