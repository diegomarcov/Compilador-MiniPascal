\documentclass[a4paper,oneside]{report}
\usepackage[spanish]{babel}
\usepackage[latin1]{inputenc}
\usepackage{fullpage}
\usepackage{listings}
\usepackage{fancyvrb}
\usepackage{float}
\usepackage[colorlinks=true,urlcolor=black,linkcolor=black]{hyperref}%

\setlength{\parskip}{1ex plus 0.5ex minus 0.2ex}

\lstset{language=,keywordstyle=\ttfamily,stringstyle=\ttfamily}
\lstset{breaklines}

\title{Compiladores e Int�rpretes\\Informe de la Segunda Entrega}

\author{Diego Marcovecchio (LU: 83815)\and Leonardo Molas (LU: 82498)}

\date{2 de Septiembre de 2010}

\begin{document}
	
\maketitle
		
\tableofcontents

\chapter*{Introducci�n}
\section*{Descripci�n}
Esta entrega consiste de un analizador l�xico para un programa de \textsc{Mini-Pascal}. Cada \textit{lexema} reconocido en el programa fuente es analizado, transformado al tipo de \textit{token} que corresponda, e impreso a la salida especificada junto con su n�mero de l�nea.

El programa tiene un nivel moderado de reconocimento de errores, permitiendo la detecci�n de errores como un comentario abierto al finalizar el archivo.

El analizador l�xico fue desarrollado utilizando �nicamente {\bf Python 2.7}\footnote{{\bf Python} es un lenguaje de programaci�n interpretado y multiplataforma. Para m�s informaci�n, dirigirse a la p�gina oficial: \url{http://www.python.org/}} y algunas de sus librer�as asociadas ({\bf re}\footnote{{\bf re} es una librer�a de Python que permite el reconocimiento de expresiones regulares. Su documentaci�n puede ser vista en: \url{http://docs.python.org/library/re.html}} y una modificaci�n propia de {\bf shlex}\footnote{{\bf shlex} es una librer�a de Python para procesar comandos de consola. Nos basamos en su c�digo fuente y realizamos algunas mejoras para procesar el stream de caracteres de entrada. La documentaci�n de la versi�n original puede ser encontrada en: \url{http://docs.python.org/library/shlex.html}}).

\chapter{Modo de uso}

\section{Requerimientos}
La versi�n ``\texttt{.exe}'' del Analizador L�xico necesita una serie de librer�as para funcionar, que fueron incluidas en la carpeta donde se encuentra el mismo ejecutable. Estas son:
\begin{itemize}\ttfamily
	\item python27.dll
	\item msvcr90.dll
	\item bz2.pyd
	\item select.pyd
	\item unicodedata.pyd
	\item library.zip \rmfamily (que contiene las librer�as de Python utilizadas)
\end{itemize}

\section{Ejecuci�n}
\lstinline[]!lexan [-h|--help] <IN_FILE> [<OUT_FILE>]!


\subsection*{Argumentos:}

\begin{description}
	\item[{\ttfamily \textless IN\_FILE\textgreater}] El archivo de Pascal de entrada.
\end{description}

\subsection*{Argumentos opcionales:}

\begin{description}
	\item[\ttfamily\textless OUT\_FILE\textgreater]  El archivo opcional de salida. 
	\item[\ttfamily -h, --help]Muestra la ayuda por pantalla.
\end{description}

Por ejemplo:

\begin{quotation}
	\lstinline!Lexan ejemplo1.pas output.txt!
\end{quotation}

En este caso, el programa leer� el archivo {\ttfamily ejemplo1.pas} y devolver� el resultado en {\ttfamily output.txt}.

\section{Formato de salida}

El Analizador L�xico devuevle la informaci�n en una tabla, donde sus columnas indican el lexema (\texttt{LEXEME}), token (\texttt{TOKEN}) y n�mero de l�nea (\texttt{LINE NUMBER}), como se puede ver en el siguiente ejemplo:

\begin{BVerbatim}[fontsize=\small ]
Starting file lexical analysis...


----------------------------------------------------------------------------------------------
|               LEXEME              |               TOKEN               |    LINE NUMBER     |
----------------------------------------------------------------------------------------------
|              program              |             <PROGRAM>             |         1          |
|              ejemplo1             |            <IDENTIFIER>           |         1          |
|                 ;                 |            <SEMI_COLON>           |         1          |
|               const               |              <CONST>              |         2          |
|                 Z                 |            <IDENTIFIER>           |         2          |
|                 =                 |              <EQUAL>              |         2          |
|                 4                 |              <NUMBER>             |         2          |
|                 ;                 |            <SEMI_COLON>           |         2          |

                                                     ...

|                 c                 |            <IDENTIFIER>           |         14         |
|                 :=                |            <ASSIGNMENT>           |         14         |
|                'd'                |               <CHAR>              |         14         |
|                 ;                 |            <SEMI_COLON>           |         14         |
|                end                |               <END>               |         16         |
|                 .                 |           <END_PROGRAM>           |         17         |
|                                   |               <EOF>               |         17         |
----------------------------------------------------------------------------------------------

Finished lexical analysis succesfully!
\end{BVerbatim}

\chapter{Lenguaje}
\section{Alfabeto de entrada}

\begin{lstlisting}
<letter>::= A | B | C | D | E | F | G | H | I | J | K | L | M | N | O | P | Q | R | S | T | U | V | W | X | Y | Z | a | b | c | d | e | f | g | h | i | j | k | l | m | n | o | p | q | r | s | t | u | v | w | x | y | z

<digit>::= 0 | 1 | 2 | 3 | 4 | 5 | 6 | 7 | 8 | 9

<special symbol> ::=  + | - | * | = | <> | < | > | <= | >= | ( | ) | [ | ] | { | } | := | . | , | ; | : | div | or | and | not | if | then | else | while | do | begin | end | const | var | type | array | function | procedure | program
\end{lstlisting}

\section{Errores detectados}
El analizador l�xico tiene un nivel moderado de detecci�n de errores. Entre �stos, se encuentran:

\begin{itemize}
	\item Caracter no reconocido: si se intenta ingresar un caracter que no pertenece al alfabeto, como ``@'', se producir� un error.
	\item Comentarios abiertos: si el programa fuente tiene un comentario sin cerrar cuando termina el archivo, se informa el error.
	\item N�meros mal formados: si se intenta ingresar un n�mero como 38a7, se informar� el error.
	\item Si el archivo fuente especificado no existe, se informar� el error.
\end{itemize}

Los mensajes de error mostrar�n el n�mero de l�nea, con el lexema que gener� el error, cuando corresponda.

\chapter{Analizador l�xico}
\section{Palabras reservadas}
El Analizador L�xico reconoce las siguientes palabras reservadas especificadas en la tabla \ref{tab:palabras}

\begin{table}[H]

\begin{tabular}{|c|}
\hline
\textbf{Palabras reservadas} \\  \hline
\ttfamily
program \\ 
\texttt{type} \\ 
\texttt{const} \\ 
\texttt{var} \\ 
\texttt{array} \\ 
\texttt{of} \\ 
\texttt{function} \\ 
\texttt{procedure} \\ 
\texttt{begin} \\ 
\texttt{end} \\ 
\texttt{while} \\ 
\texttt{do} \\ 
\texttt{if} \\ 
\texttt{then} \\ 
\texttt{else} \\ 
\texttt{div} \\ 
\texttt{not} \\ 
\texttt{or} \\ 
\texttt{and} \\ 
\texttt{true} \\ 
\texttt{false} \\ 
\hline
\end{tabular}
\label{tab:palabras}
\caption{Palabras reservadas}
\end{table}
\section{Tokens}
El Analizador L�xico reconoce los tokens especificados en la tabla \ref{tab:tokens}.
\begin{table}[H]

\begin{tabular}{|l|l|}
\hline
\textbf{Token}			& \textbf{Expresi�n Regular} \\ \hline
Identifier 					& {\ttfamily [a-zA-Z][a-zA-Z0-9]*} \\ \hline
Number							& {\ttfamily [0-9]+} \\ \hline
Char								& {\ttfamily '[a-zA-Z0-9]'} \\ \hline
RelOp      					& {\ttfamily <|>|<=|>=} \\ \hline
Arith\_Op 					& {\ttfamily +|-|*}\\ \hline
Un\_LogOp 					& {\ttfamily not} \\ \hline
Bin\_LogOp 					& {\ttfamily or|and} \\ \hline
Equal 							& {\ttfamily =} \\ \hline
Type\_Declaration 	& {\ttfamily :} \\ \hline
Assignment 					& {\ttfamily :=} \\ \hline
Comma 							& {\ttfamily ,} \\ \hline
Semicolon 					& {\ttfamily ;} \\ \hline
End\_Program			 	& {\ttfamily .} \\ \hline
Subrange\_Separator & {\ttfamily ..} \\ \hline
EOF 								&  \\ \hline
Open\_Parenthesis 	& {\ttfamily (} \\ \hline
Close\_Parenthesis 	& {\ttfamily )} \\ \hline
Open\_Bracket 			& {\ttfamily [} \\ \hline
Close\_Bracket 			& {\ttfamily ]} \\ \hline
Program 						& {\ttfamily program} \\ \hline
Type 								& {\ttfamily type} \\ \hline
Const 							& {\ttfamily const} \\ \hline
Var 								& {\ttfamily var} \\ \hline
Function 						& {\ttfamily function} \\ \hline
Procedure 					& {\ttfamily procedure} \\ \hline
Array 							& {\ttfamily array} \\ \hline
Of 									& {\ttfamily of} \\ \hline
Begin 							& {\ttfamily begin} \\ \hline
End 								& {\ttfamily end} \\ \hline
While 							& {\ttfamily while} \\ \hline
Do 									& {\ttfamily do} \\ \hline
If 									& {\ttfamily if} \\ \hline
Then 								& {\ttfamily then} \\ \hline
Else 								& {\ttfamily else} \\ \hline
\end{tabular}
\label{tab:tokens}
\caption{Tokens}
\end{table}


\end{document}