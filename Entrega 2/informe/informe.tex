\documentclass[a4paper,oneside]{report}
\usepackage[spanish]{babel}
\usepackage[latin1]{inputenc}
\usepackage{graphicx}
\usepackage{latexsym}
\usepackage{fullpage}
\usepackage{float}

\setlength{\parskip}{1ex plus 0.5ex minus 0.2ex}

\title{Compiladores e Int�rpretes\\Especificaci�n del Lenguaje}

\author{Diego Marcovecchio (LU: 83815)\and Leonardo Molas (LU: 82498)}

\begin{document}
	

\maketitle
		
\chapter*{Introducci�n}
\section*{Descripci�n}
Esta entrega consiste de un analizador l�xico para un programa de mini-pascal. Cada lexema reconocido en el programa fuente es analizado, transformado al tipo de token que corresponda, e impreso a la salida especificada junto con su n�mero de l�nea.

El programa tiene un nivel moderado de reconocimento de errores, permitiendo la detecci�n de errores como un comentario abierto al finalizar el archivo.

El analizador l�xico fue desarrollado utilizando �nicamente {\bf Python 2.7}\footnote{{\bf Python} es un lenguaje de programaci�n interpretado y multiplataforma. Para m�s informaci�n, dirigirse a la p�gina oficial: \url{http://www.python.org/}} y algunas de sus librer�as asociadas ({\bf re}\footnote{{\bf re} es una librer�a de Python que permite el reconocimiento de expresiones regulares. Su documentaci�n puede ser vista en: \url{http://docs.python.org/library/re.html}} y una modificaci�n propia de {\bf shlex}\footnote{{\bf shlex} es una librer�a de Python para procesar comandos de consola. Nos basamos en su c�digo fuente y realizamos algunas mejoras para procesar el stream de caracteres de entrada. La documentaci�n de la versi�n original puede ser encontrada en: \url{http://docs.python.org/library/shlex.html}}).

\section*{Modo de uso}
\begin{tabbing}
\hspace*{2cm}\=\hspace{2.2cm}\=\hspace{6cm}\= \kill
LexAn [-h] \> \textless IN\_FILE\textgreater \> [\textless OUT\_FILE\textgreater]\\\\
\end{tabbing}



Argumentos:

\begin{tabbing}
\hspace*{3cm}\=\hspace{3cm}\=\hspace{6cm}\= \kill
\>\textless IN\_FILE\textgreater \> El archivo de pascal de entrada.\>\\
\end{tabbing}

Argumentos opcionales:

\begin{tabbing}
\hspace*{3cm}\=\hspace{3cm}\=\hspace{6cm}\= \kill
\>\textless OUT\_FILE\textgreater\> El archivo opcional de salida.\>\\
\>-h, --help\>Muestra la ayuda por pantalla. \\
\end{tabbing}
\chapter{Analizador l�xico}
\end{tabbing}
\begin{table}[htbp]

\begin{tabular}{|l|l|}
\hline
Identifier & (letter)(letter/digit)* \\ \hline
Number & (digit)(digit)* \\ \hline
RelOp & \textless, \textgreater, \textless\textgreater, \textless=, \textgreater= \\ \hline
Arith\_Op & + | - \\ \hline
Un\_LogOp & not \\ \hline
Bin\_LogOp & or | and \\ \hline
Equal & = \\ \hline
Type\_Declaration & : \\ \hline
Assignment & := \\ \hline
Comma & , \\ \hline
Semicolon & ; \\ \hline
End\_Program & . \\ \hline
Subrange\_Separator & .. \\ \hline
EOF & NULL \\ \hline
Open\_Parenthesis & ( \\ \hline
Close\_Parenthesis & ) \\ \hline
Open\_Bracket & [ \\ \hline
Close\_Bracket & ] \\ \hline
Program & program \\ \hline
Type & type \\ \hline
Const & const \\ \hline
Var & var \\ \hline
Function & function \\ \hline
Procedure & procedure \\ \hline
Array & array \\ \hline
Of & of \\ \hline
Begin & begin \\ \hline
End & end \\ \hline
While & while \\ \hline
Do & do \\ \hline
If & if \\ \hline
Then & then \\ \hline
Else & else \\ \hline
\end{tabular}
\label{}
\caption{Tokens}
\end{table}


\end{document}