\documentclass[a4paper,oneside]{report}
\usepackage[spanish]{babel}
\usepackage[latin1]{inputenc}
\usepackage{fullpage}
\usepackage[colorlinks=true,urlcolor=black,linkcolor=black]{hyperref}%

\setlength{\parskip}{1ex plus 0.5ex minus 0.2ex}


\title{Compiladores e Int�rpretes\\Bater�a de ejemplos}

\author{Diego Marcovecchio (LU: 83815)\and Leonardo Molas (LU: 82498)}

\date{2 de Septiembre de 2010}

\begin{document}
	
\maketitle

\chapter*{Casos de prueba}

Los siguientes son los casos de prueba entregados para ser ejecutados junto con el programa, con sus respectivos comportamientos.

\section*{Casos de prueba correctos}
\subsection*{\texttt{comentarios.pas}}
Se declaran tipos, constante, y variables. En el cuerpo del programa hay asignaciones de muchos tipos. Tambi�n se utilizan distintos tipos de comentarios, que son obviados.

\subsection*{\texttt{proc\_y\_fun.pas}}
Se declaran funciones, procedimientos, y variables. Se producen asignaciones y llamadas.

\subsection*{\texttt{comentarios2.pas}}
Se declaran tipos, y variables. Hay m�s comentarios, con casos m�s extremos que el anterior.

\subsection*{\texttt{llamada\_proc.pas}}
Se declaran funci�n, procedimiento, tipo y variables. Se realizan asignaciones, llamadas, y utilizaci�n de estructuras de control. Tambi�n hay comentarios.

\subsection*{\texttt{ejemplo\_full.pas}}
Ejemplo con todas las estructuras y tokens posibles puestos juntos. Se trata de la implementaci�n de un Fibonacci en \textsc{Mini-Pascal}.

\section*{Casos de prueba incorrectos}
\subsection*{\texttt{caracter\_erroneo.pas}}

En el cuerpo del programa se utiliza un caracter que no pertece al alfabeto reconocido por el lenguaje.

\subsection*{\texttt{comentario\_abierto.pas}}
Se deja un comentario abierto, por lo que el analizador llega a \texttt{EOF} antes de encontrar el terminador.

\subsection*{\texttt{numero\_erroneo.pas}}
Un n�mero es formado de una manera err�nea en una asignaci�n, generando el error.


\end{document}